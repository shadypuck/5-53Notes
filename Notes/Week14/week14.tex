\documentclass[../notes.tex]{subfiles}

\pagestyle{main}
\renewcommand{\chaptermark}[1]{\markboth{\chaptername\ \thechapter\ (#1)}{}}
\setcounter{chapter}{13}

\begin{document}




\chapter{Exotic Reactivity}
\section{Kinetic Resolution and Related Asymmetric Processes}
\begin{itemize}
    \item \marginnote{12/3:}Announcements.
    \begin{itemize}
        \item Today: Last Tuesday's lecture.
        \begin{itemize}
            \item Next time: Electron Transfer.
        \end{itemize}
        \item Exam 2 tomorrow.
        \begin{itemize}
            \item Format like the practice exam.
            \item Administered remotely.
            \item Work alone, and closed note (honor code).
            \item Available for 48 hours: Start of Wednesday til end of Thursday.
        \end{itemize}
        \item Do the teaching evaluations for both Alex and Masha!!
    \end{itemize}
    \item Today: Kinetic selectivities.
    \item Consider a starting material (\ce{SM}) that can evolve to a product \ce{A} or \ce{B}.
    \begin{figure}[h!]
        \centering
        \includegraphics[width=0.55\linewidth]{selectThermKin.JPG}
        \caption{Thermodynamic vs. kinetic selectivity energy diagram.}
        \label{fig:selectThermKin}
    \end{figure}
    \begin{itemize}
        \item We can map this reaction onto a potential energy surface.
        \item If \ce{A} and \ce{B} are free to reversibly interconvert, then we can explain the product distribution in terms of the $\Delta G^\circ$ between \ce{A} and \ce{B}.
        \begin{itemize}
            \item In particular, $\Delta G=-RT\ln\Keq$ where $\Keq=\cnc{A}/\cnc{B}$.
        \end{itemize}
        \pagebreak
        \item Today, we'll consider the case in which \ce{A} and \ce{B} do \emph{not} reversibly interconvert.
        \begin{itemize}
            \item In this case, what's important is the $\Delta\Delta G^\ddagger$ between the transition states.
            \item Here, the selectivity is given as the ratio of the rate constants:
            \begin{equation*}
                \text{selectivity} = \frac{\cnc{A}}{\cnc{B}}
                = \frac{k_{\ce{A}}}{k_{\ce{B}}}
                = \frac{\e[-\Delta G^\ddagger_{\ce{A}}/RT]}{\e[-\Delta G^\ddagger_{\ce{B}}/RT]}
                = \e[-\Delta\Delta G^\ddagger/RT]
            \end{equation*}
            \item Note that $k_{\ce{A}}/k_{\ce{B}}=\krel$. This quantity is important for determing dr's, er's, etc.
        \end{itemize}
    \end{itemize}
    \item A case in which kinetic selectivity is important: Catalytic kinetic resolution.
    \begin{figure}[h!]
        \centering
        \includegraphics[width=0.4\linewidth]{EDcatKinRes.JPG}
        \caption{Catalytic kinetic resolution energy diagram.}
        \label{fig:EDcatKinRes}
    \end{figure}
    \begin{itemize}
        \item \ce{A} is our starting material, a chiral racemic compound.
        \begin{itemize}
            \item Thus, we can denote the starting materials as \ce{A_S + A_R}.
            \item As enantiomers, \ce{A_R} and \ce{A_S} have identical free energies.
        \end{itemize}
        \item \ce{cat^*} is a \textbf{homochiral} catalyst.
        \item Then resolution to a product \ce{B} can happen two different ways: Through a transition state that consumes the (\emph{S})-enantiomer, and through a transition state that consumes the (\emph{R})-enantiomer.
        \item When a homochiral catalyst acts on two enantiomers, it forms two different, diastereomeric adducts: \ce{A_S*cat^*} and \ce{A_R*cat^*}.
        \begin{itemize}
            \item Unlike enantiomers, diastereomers \emph{are} different compounds that may have two different energies.
        \end{itemize}
        \item What we've indicated in Figure \ref{fig:EDcatKinRes} is that the (\emph{S})-enantiomer is converted faster than the (\emph{R})-enantiomer.
        \item Thus,
        \begin{equation*}
            \krel = \frac{\kf}{\ks}
            = \e[-\Delta\Delta G^\ddagger/RT]
        \end{equation*}
        \begin{itemize}
            \item In the literature, $\krel$ is sometimes referred to as an \textbf{S-factor} (for "selectivity factor").
        \end{itemize}
        \item Reference: \textcite{bib:EDcatKinRes}.
    \end{itemize}
    \item \textbf{Homochiral} (catalyst): A chiral catalyst of which we're using only a single enantiomer.
    \item There are many catalytic kinetic resolutions in the literature.
    \begin{itemize}
        \item Radosevich developed one in grad school, when he was roughly our age!
    \end{itemize}
    \pagebreak
    \item Example catalytic kinetic resolution.
    \begin{figure}[h!]
        \centering
        \footnotesize
        \schemestart
            \chemname{
                \chemfig{*3([7](-Me)--O-)}
            }{($\pm$)}
            \+
            \chemname{
                \chemfig{H_2O}
            }{0.55 eq.}
            \arrow(.7--){->[0.2 mol\% (\emph{R},\emph{R})-\ce{(salen)Co^{III}(OAc)}][neat]}[,3.6]
            \chemnameinit{}
            \chemname{
                \chemfig{Me-[:15]*3(--O>:)}
            }{\tikz{\node[align=center]{44\% yield\\99\% ee};}}
            \+{,,0.7em}
            \chemname{
                \chemfig{Me-[:30](<[2]OH)-[:-30]-[:30]OH}
            }{\tikz{\node[align=center]{50\% yield\\98\% ee};}}
        \schemestop
        \caption{Hydrolytic kinetic resolution.}
        \label{fig:kinResHydro}
    \end{figure}
    \begin{itemize}
        \item Take propylene oxide (racemic) and 0.55 eq. of water.
        \item React them, neat, in the presence of a small amount of homochiral catalyst.
        \begin{itemize}
            \item The structure of this complex is totally irrelevant to our aims, but we can look it up in the reference if we're curious.
            \item This is our homochiral catalyst that will act on the two relatively inexpensive starting materials.
        \end{itemize}
        \item We run this reaction neat, and recover one enantiomer of our starting material in nearly quantitative yield with near perfect ee.
        \item We also obtain a ring-opened \emph{vic}-diol in nearly quantitative yield with near perfect ee.
        \item $\krel\approx 500$ here!
        \item Reference: \textcite{bib:kinResHydro}.
        \item This is a \textbf{hydrolytic kinetic resolution}.
        \begin{itemize}
            \item This is a very useful reaction for the resolution of terminal epoxides --- and access to terminal 1,2-diols --- because there exists no method to synthetically prefer a single enantiomer.
            \item Propylene is so small that even the best chiral epoxidation catalysts aren't very selective here, so it's better to do a racemic epoxidation and then this.
        \end{itemize}
        \item Great atom economy.
    \end{itemize}
    \item In a kinetic resolution like the above, the percent ee of both starting material and product is subject to change over time.
    \item To see this, let's build a theoretical model for a catalytic kinetic resolution.
    \begin{align*}
        \ce{A_S + cat^*} &\ce{->[$\kf$] P}&
        \ce{A_R + cat^*} &\ce{->[$\ks$] P}
    \end{align*}
    \begin{itemize}
        \item The net transformation involves the above two chemical reactions.
        \item We can write differential rate laws for each enantiomer
        \begin{align*}
            \dv{\cnc{A_S}}{t} &= -\kf\cnc{A_S}\cnc{cat^*}&
            \dv{\cnc{A_R}}{t} &= -\ks\cnc{A_R}\cnc{cat^*}
        \end{align*}
        \begin{itemize}
            \item The consumption of the fast-reaction enantiomer will deplete $\cnc{A}$.
        \end{itemize}
        \item Assuming $\cnc{cat^*}$ is approximately constant throughout the reaction, each of these differential rate laws can be independently integrated to
        \begin{align*}
            \ln(\frac{\cnc{A_S}}{\cnc[0]{A_S}}) &= -\kf\cnc{cat^*}t&
            \ln(\frac{\cnc{A_R}}{\cnc[0]{A_R}}) &= -\ks\cnc{cat^*}t
        \end{align*}
        \item Then, the key thing to note here is that for a racemic mixture, $\cnc[0]{A_S}=\cnc[0]{A_R}$.
        \item Thus, we can make this substitution and divide the above two integrated rate laws to get
        \begin{equation*}
            \krel = \frac{\kf}{\ks}
            = \frac{\ln(\cnc{A_S}/\cnc[0]{A_S})}{\ln(\cnc{A_R}/\cnc[0]{A_S})}
        \end{equation*}
    \end{itemize}
    \item This is a useful result, but we can make it even better. We'll start with a couple of definitions.
    \item \textbf{Conversion}: The ratio of how much of a reactant has reacted. \emph{Denoted by} $\bm{c}$. \emph{Given by}
    \begin{equation*}
        c := 1-\frac{\cnc{A_S}+\cnc{A_R}}{\cnc[0]{A_S}+\cnc[0]{A_R}}
        = 1-\frac{\cnc{A_S}+\cnc{A_R}}{2\cnc[0]{A_S}}
    \end{equation*}
    \item \textbf{Enantiomeric excess}: A measurement of the degree to which a sample contains one enantiomer in greater amounts than the other. \emph{Denoted by} \textbf{ee}. \emph{Given by}
    \begin{equation*}
        \text{ee} := \frac{\cnc{A_S}-\cnc{A_R}}{\cnc{A_S}+\cnc{A_R}}
    \end{equation*}
    \item We can now do some algebra.
    \begin{itemize}
        \item Indeed, it follows from the above definitions that
        \begin{align*}
            1-\text{ee} &= \frac{2\cnc{A_R}}{\cnc{A_S}+\cnc{A_R}}&
            1+\text{ee} &= \frac{2\cnc{A_S}}{\cnc{A_S}+\cnc{A_R}}
        \end{align*}
        \item Then we can derive the following interesting relatinoships.
        \begin{align*}
            \frac{\cnc{A_R}}{\cnc[0]{A_S}} &= (1-c)(1-ee)&
            \frac{\cnc{A_S}}{\cnc[0]{A_S}} &= (1-c)(1+ee)
        \end{align*}
        \item We can now know the extent to which a reaction has evolved to consume one enantiomer or the other as a function of observables!
    \end{itemize}
    \item Thus, we may define $\text{S}=\krel$ as a function of conversion and ee.
    \begin{itemize}
        \item For recovered starting material,
        \begin{equation*}
            \text{S} = \frac{\ln[(1-c)(1-ee)]}{\ln[(1-c)(1+ee)]}
        \end{equation*}
        \item For the product,
        \begin{equation*}
            \text{S} = \frac{\ln[(1-c)(1+ee)]}{\ln[(1-c)(1-ee)]}
        \end{equation*}
    \end{itemize}
    \item These relations allow us to relate conversion to ee for a catalyst of a given, set selectivity S. Specifically, we can parametrically plot ee as a function of conversion.
    \item Let's first do this for the percent ee in the recovered starting material.
    \begin{figure}[H]
        \centering
        \includegraphics[width=0.45\linewidth]{kinResSMee.JPG}
        \caption{Starting material ee vs. conversion in a catalytic kinetic resolution.}
        \label{fig:kinResSMee}
    \end{figure}
    \begin{itemize}
        \item Consider first what happens in the limit that our selectivity factor is very large, i.e., that $\Delta\Delta G^\ddagger=\text{large}$.
        \item If $S=\infty$, then 50\% conversion will get us all we need.
        \begin{itemize}
            \item This is because at this point, the enantiomer we don't want to recover will have been fully consumed.
        \end{itemize}
        \item As the S-factor drops, we need higher conversions to get better ee's in the recovered starting material.
        \item What's cool about this is we can still get high ee's with bad catalysts\dots at the expense of conversion.
        \begin{itemize}
            \item Essentially, with bad catalysts, we'll recover \emph{less} enantiopure starting material (because some of it will have been consumed at higher conversions), but we \emph{can} still recover essentially enantiopure starting material.
        \end{itemize}
    \end{itemize}
    \item For the product.
    \begin{figure}[h!]
        \centering
        \includegraphics[width=0.43\linewidth]{kinResPee.JPG}
        \caption{Product ee vs. conversion in a catalytic kinetic resolution.}
        \label{fig:kinResPee}
    \end{figure}
    \begin{itemize}
        \item If $S=\infty$, the product will be enantiopure up until we begin converting some of the other enantiomer.
        \item If $S=50$, we start at near-optimal purity, and then our bias will erode.
        \item What this implies is that for the purpose of kinetic resolution of the product, we need very good catalysts.
        \item That's what's remarkable about the Jacobsen catalyst: It's extremely selective for both the starting material \emph{and} product.
    \end{itemize}
    \item Let's now enter into some more complex kinetic regimes.
    \begin{figure}[h!]
        \centering
        \begin{subfigure}[b]{0.25\linewidth}
            \centering
            \includegraphics[width=0.7\linewidth]{kinResDyna.JPG}
            \caption{Typical setup.}
            \label{fig:kinResDyna}
        \end{subfigure}
        \begin{subfigure}[b]{0.25\linewidth}
            \centering
            \includegraphics[width=0.65\linewidth]{kinResDynb.JPG}
            \caption{Direct interconversion.}
            \label{fig:kinResDynb}
        \end{subfigure}
        \begin{subfigure}[b]{0.25\linewidth}
            \centering
            \includegraphics[width=0.7\linewidth]{kinResDync.JPG}
            \caption{Achiral intermediate.}
            \label{fig:kinResDync}
        \end{subfigure}
        \caption{Dynamic kinetic resolution models.}
        \label{fig:kinResDyn}
    \end{figure}
    \pagebreak
    \begin{itemize}
        \item As we've depicted it in Figure \ref{fig:EDcatKinRes}, our starting materials are equal in energy and not interconverting.
        \begin{itemize}
            \item We can conceptualize this scenario as having a mirror plane between our starting materials and products that we \emph{never} cross (Figure \ref{fig:kinResDyna}).
        \end{itemize}
        \item But what about when the starting materials do interconvert?
        \begin{itemize}
            \item There are two ways in which this can happen: We can cross the mirror plane directly (Figure \ref{fig:kinResDynb}), or through an achiral intermediate (Figure \ref{fig:kinResDync}).
        \end{itemize}
        \item The 50\% mass balance limit that is otherwise imposed is now lifted!
        \item Now our catalyst can sample both enantiomers via the epimerization.
        \item \ce{A_S} interconverts with \ce{A_R}, subject to a kinetically selective catalyst.
    \end{itemize}
    \item Enantiomers under fast equilibrium are still under kinetic control, but with 100\% theoretical yield.
    \begin{figure}[h!]
        \centering
        \begin{tikzpicture}[
            every node/.style=black
        ]
            \footnotesize
            \draw (0,4) -- (0,0) -- (6,0);
    
            \draw [dashed] (0.9,1.9) -- (4.7,1.9);
            \draw [->,shorten >=1pt] (1.1,1.9) -- node[fill=white,inner sep=1pt]{$\Delta G^\ddagger_{\ce{S}}$} (1.1,3);
            \draw [->,shorten >=1pt] (4.5,1.9) -- node[fill=white,inner sep=1pt]{$\Delta G^\ddagger_{\ce{R}}$} (4.5,3.8);
    
            \draw [grx,thick] (0.25,0.5) node[below]{\ce{P_S}}
                to[out=0,in=180,looseness=0.6] (1.1,3)
                to[out=0,in=180,out looseness=0.6] (2.1,1.9) node[below]{\ce{A_S}}
                to[out=0,in=180] (2.8,2.2)
                to[out=0,in=180] (3.5,1.9) node[below]{\ce{A_R}}
                to[out=0,in=180,in looseness=0.7] (4.5,3.8)
                to[out=0,in=180,looseness=0.6] (5.75,0.5) node[below]{\ce{P_R}}
            ;
        \end{tikzpicture}
        \caption{Dynamic kinetic resolution energy diagram.}
        \label{fig:EDkinResDyn}
    \end{figure}
    \begin{itemize}
        \item This is known as a \textbf{dynamic kinetic resolution}.
    \end{itemize}
    \item Example dynamic kinetic resolution: Interconversion of chiral $\beta$-ketoesters prior to asymmetric hydrogenation.
    \begin{figure}[H]
        \centering
        \footnotesize
        \schemestart
            \chemfig{Me-[:30](=[2]O)-[:-30](<[6]-[:-30]NHBz)-[:30]CO_2Me}
            \arrow(--.174){->[\tikz{\node[align=center]{(\emph{R})-BINAP-Ru\\\ce{H2}, base}}][$k_\text{fast}$]}[,1.9]
            \chemname{
                \chemfig{Me-[:30](<[2]OH)-[:-30](<[6]-[:-30]NHBz)-[:30]CO_2Me}
            }{\tikz{\node[align=center]{94\% yield\\99\% ee}}}
            \+{,,0.4em}
            \chemname{
                \chemfig{Me-[:30](<:[2]OH)-[:-30](<[6]-[:-30]NHBz)-[:30]CO_2Me}
            }{0.1\% yield}
            \arrow(@c1--){<=>}[-135]
            \chemfig{Me-[:30](-[2]OH)=^[:-30](-[6]-[:-30]NHBz)-[:30]CO_2Me}
            \arrow{<=>}[-45]
            \chemfig{Me-[:30](=[2]O)-[:-30](<:[6]-[:-30]NHBz)-[:30]CO_2Me}
            \arrow(--.176){->[\tikz{\node[align=center]{(\emph{R})-BINAP-Ru\\\ce{H2}, base}}][$k_\text{slow}$]}[,1.9]
            \chemname{
                \chemfig{Me-[:30](<[2]OH)-[:-30](<:[6]-[:-30]NHBz)-[:30]CO_2Me}
            }{5.6\% yield}
            \+{,,0.5em}
            \chemname{
                \chemfig{Me-[:30](<:[2]OH)-[:-30](<:[6]-[:-30]NHBz)-[:30]CO_2Me}
            }{0.5\% yield}
        \schemestop
        \caption{Noyori asymmetric hydrogenation.}
        \label{fig:noyoriAsym}
    \end{figure}
    \begin{itemize}
        \item Consider a racemic sample of $\alpha$-alkylated $\beta$-ketoester.
        \item Subject it to hydrogenation under \textbf{Noyori conditions}.
        \begin{itemize}
            \item Both enantiomeric starting materials may become a \emph{syn} or \emph{anti} diastereomers.
            \item Thus, in principle, you'd get a mess.
        \end{itemize}
        \item Our mess is slightly alleviated by the fact that $\kf/\ks=15$ for the ruthenium-BINAP catalyst.
        \begin{itemize}
            \item This is about \kcal{2} of difference.
            \item However, per Figure \ref{fig:kinResPee}, this S-factor is not great.
        \end{itemize}
        \item Our saving grace is the dynamic nature of this reaction.
        \begin{itemize}
            \item When we actually run the experiment, we get fast enolization and interconversion through an achiral intermediate because of the base\footnote{It appears from the paper that there may not be a base; this may just be keto-enol tautomerization.} in solution and the acidic $\alpha$-proton.
            \item Indeed, if the rate of enolization/racemation is denoted by $k_\text{rac}$, we have $k_\text{rac}/\kf\approx 100$!
        \end{itemize}
        \item Thus, we get 94\% yield of one stereoisomer in 99\% ee.
        \item Reference: \textcite{bib:noyoriAsym}.
        \begin{itemize}
            \item See Table 3, Figure 19, and the associated discussions.
            \item The whole paper is a good review of this chemistry, though.
        \end{itemize}
    \end{itemize}
    \item \textbf{Noyori asymmetric hydrogenation}: The asymmetric hydrogenation of a ketone using a homochiral ruthenium-BINAP catalyst, hydrogen gas, and a base.
    \item A related kinetic selectivity: Curtin-Hammett kinetics.
    \begin{figure}[h!]
        \centering
        \includegraphics[width=0.4\linewidth]{CHselectDeriv.JPG}
        \caption{Curtin-Hammett selectivity derivation.}
        \label{fig:CHselectDeriv}
    \end{figure}
    \begin{itemize}
        \item Instead of (\emph{R})- and (\emph{S})-enantiomers, which rigorously have the same energy, we can consider other interconverting species with different energies.
        \item We can quantitate --- with rate laws --- the formation of the products.
        \begin{align*}
            \dv{\cnc{P1}}{t} &= k_1\cnc{S1}&
            \dv{\cnc{P2}}{t} &= k_2\cnc{S2}
        \end{align*}
        \item Then taking a ratio gives
        \begin{equation*}
            \dv{\cnc{P1}}{\cnc{P2}} = \frac{k_1\cnc{S1}}{k_2\cnc{S2}}
        \end{equation*}
        \item Note that
        \begin{equation*}
            \frac{\cnc{S1}}{\cnc{S2}} = \Keq
        \end{equation*}
        so
        \begin{equation*}
            \dv{\cnc{P1}}{\cnc{P2}} = \frac{k_1}{k_2}\Keq
        \end{equation*}
        \item Thus, with Arrhenius,
        \begin{equation*}
            \dv{\cnc{P1}}{\cnc{P2}} = \frac{\e[-\Delta G^\ddagger_2/RT]}{\e[-\Delta G^\ddagger_1/RT]}\e[-\Delta G^\circ/RT]
            = \e[(\Delta G^\ddagger_1-\Delta G^\ddagger_2-\Delta G^\circ)/RT]
        \end{equation*}
        \item And then referencing Figure \ref{fig:CHselectDeriv}, we can see pictorially that
        \begin{equation*}
            \Delta G^\ddagger_1-\Delta G^\ddagger_2-\Delta G^\circ = \Delta\Delta G^\ddagger
        \end{equation*}
        \item Thus, 
        \begin{equation*}
            \dv{\cnc{P1}}{\cnc{P2}} = \e[\Delta\Delta G^\ddagger/RT]
        \end{equation*}
        if we have a fast equilibrium \ce{S1 <=> S2} (10 times faster than \ce{P1} or \ce{P2} formation).
        \item Essentially, if we have this fast starting equilibrium, then the product ratio is under kinetic control.
    \end{itemize}
    \item Now suppose we drop \ce{S2} down in free energy and leave the rest of the diagram unperturbed.
    \begin{itemize}
        \item This change in one variable is compensated for by a change in the other variable, and we remain under kinetic control.
    \end{itemize}
    \item Alex briefly discusses kinetic quench.
    \item Curtin-Hammett example 1 (Figure \ref{fig:CHscenarioa}).
    \begin{itemize}
        \item \ce{P1} is kinetically favored.
        \item $\cnc{S1}>\cnc{S2}$.
        \item Here, the \ce{S1}/\ce{S2} ratio is irrelevant to product formation. This is "invisible" C/H kinetics. Mathematically,
        \begin{equation*}
            \frac{\cnc{S1}}{\cnc{S2}} \neq \frac{\cnc{P1}}{\cnc{P2}}
        \end{equation*}
    \end{itemize}
    \item Curtin-Hammett example 2 (Figure \ref{fig:CHscenariob}).
    \begin{itemize}
        \item \ce{P1} is kinetically favored.
        \item $\cnc{S1}<\cnc{S2}$.
        \item This is "classic" C/H kinetics.
        \item Great example of this in \textcite{bib:CHexample}.
        \item This scenario is actually pretty common.
    \end{itemize}
    \item Curtin-Hammett example 3 (Figure \ref{fig:CHscenarioc}).
    \begin{itemize}
        \item Here, $\Delta G^\ddagger_1=\Delta G^\ddagger_2$.
        \item This scenario is pretty uncommon, but it is possible.
        \item In this case, the equilibrium ratio \emph{does} reflect the product ratio.
    \end{itemize}
    \item Takeaway: It is far more likely that your equilibrium ratio of intermediates has no bearing on your ratio of products.
\end{itemize}



\section{Exam 2 Review Sheet}
\begin{itemize}
    \item \marginnote{12/4:}Noncovalent interactions.
    \begin{itemize}
        \item Quadrupoles preferentially bind hard, positively charged ions.
        \item The strongest hydrogen bonds have short bond lengths and are nearly linear; vice versa for weak.
        \item There is an energetic penalty to mixing polar and nonpolar solvents. The penalty is mostly entropic, because putting a hydrophobic link in the water disrupts the water's ability to randomly hydrogen bond to itself. Less \ce{H}-bonding means more ordered, less entropic water.
    \end{itemize}
    \item Transition states are first-order saddle points on a hypersurface defined by vibrational DOFs.
    \begin{itemize}
        \item $\Delta G=-RT\ln\Keq$.
        \item $\Delta G^\ddagger=-RT\ln k$.
        \item $\Delta\Delta G^\ddagger=-RT\ln(k_{\ce{A}}/k_{\ce{B}})$
    \end{itemize}
    \item \textbf{van't Hoff analysis}: Experimental determination of $\Delta H^\circ$ and $\Delta S^\circ$.
    \begin{align*}
        -RT\ln\Keq &= \Delta H^\circ-T\Delta S^\circ\\
        R\ln\Keq &= -\Delta H^\circ\left( \frac{1}{T} \right)+\Delta S^\circ
    \end{align*}
    \item Fragmentations give \kcal{9}, or \eu{30} at \SI{300}{\kelvin}.
    \item Transition state theory.
    \begin{itemize}
        \item General form.
        \begin{equation*}
            \ce{A <=>[$K^\ddagger$] [TS] ->[$k^\ddagger$] B}
        \end{equation*}
        \item Postulates.
        \begin{enumerate}
            \item Activated complex is in quasi-equilibrium with starting material.
            \item Any molecule that makes its way to the transition state will then proceed onto the product barrierlessly ($\kappa=1$).
        \end{enumerate}
        \item The number of times a starting material appears in the TS is it's order in the rate law.
        \item Eyring equation:
        \begin{equation*}
            k = \kappa\left( \frac{\kB T}{h} \right)\e[-\Delta G^\ddagger/RT]
        \end{equation*}
        \begin{itemize}
            \item Linearization (for experimental determination of $\Delta H^\ddagger$ and $\Delta S^\ddagger$):
            \begin{equation*}
                \ln(\frac{kh}{\kappa\kB T}) = -\frac{\Delta H^\ddagger}{R}\left( \frac{1}{T} \right)+\frac{\Delta S^\ddagger}{R}
            \end{equation*}
        \end{itemize}
        \item Hammond postulate.
    \end{itemize}
    \item Isotope effects.
    \begin{itemize}
        \item QMech foundation:
        \begin{align*}
            E_n &= h\nu\left( n+\frac{1}{2} \right)&
            \nu &= \frac{1}{2\pi}\sqrt{\frac{k}{\mu}}
        \end{align*}
        \item \ce{C-D} lower by \kcal{1.5}.
        \item Equilibrium isotope effect:
        \begin{equation*}
            \frac{K_{\ce{H}}}{K_{\ce{D}}}
        \end{equation*}
        \item \ce{C-D} bonds hold onto their electrons more tightly.
        \item Kinetic isotope effect.
        \begin{equation*}
            \KIE = \frac{k_{\ce{H}}}{k_{\ce{D}}}
        \end{equation*}
        \item Asymmetric stretch is the reaction coordinate.
        \begin{itemize}
            \item Symmetric stretch is the important orthogonal vector.
            \item Under thermoneutral conditions with identical atoms on either side, $\KIE=\max$ because the symmetric stretch has no isotopic sensitivity.
        \end{itemize}
        \item Bent transition states have more contributions from bending mode isotopic sensitivities, but smaller overall sensitivities (1.5-3.5).
        \item Quantum tunnelling can give anomously large KIEs.
        \item Secondary isotope effects at the $\alpha$-position largely governed by out-of-plane bending modes.
        \begin{itemize}
            \item Going to a slacker potential: Normal $2^\circ$ KIE.
            \item Going to a stiffer potential: Inverse $2^\circ$ KIE.
        \end{itemize}
        \item Secondary isotope effects at the $\beta$-position governed by hyperconjugation, or actually primary because involved (e.g., E\textsubscript{2}).
        \item Experimental determination.
        \begin{itemize}
            \item Independent absolute rate measurement.
            \item Intramolecular competition experiment.
            \begin{itemize}
                \item Correction: If
                \begin{align*}
                    C &:= \frac{\cnc{P_H}}{\cnc[0]{SM_H}}&
                    R &:= \left( \frac{\cnc{SM_D}}{\cnc{SM_H}} \right)_t&
                    R_0 &:= \left( \frac{\cnc{SM_D}}{\cnc{SM_H}} \right)_0
                \end{align*}
                then
                \begin{equation*}
                    \KIE = \frac{k_{\ce{H}}}{k_{\ce{D}}}
                    = \frac{\ln(1-C)}{\ln\left[ (1-C)\cdot\frac{R}{R_0} \right]}
                \end{equation*}
            \end{itemize}
            \item Intramolecular competition can probe post-rate-determining steps.
            \begin{itemize}
                \item Extract KIE from the product ratio at \emph{any} conversion.
            \end{itemize}
        \end{itemize}
        \item Heavy-atom KIEs.
        \begin{itemize}
            \item Extracted at high conversions.
            \item Choose a reference atom, take NMRs at high conversion, plug into conversion formula to get KIEs.
            \item Can tell you what sites are involved in the RDS!
            \item Conversion-\emph{dependent} isotopic enrichment is affiliated with the RDS.
            \item Conversion-\emph{independent} isotopic enrichment is affiliated with post-RDS steps.
        \end{itemize}
    \end{itemize}
    \item Rate laws.
    \begin{itemize}
        \item Zeroeth-order, first-order, and second-order integrated rate laws.
        \item $\kobs$ vs. swamping concentrations to determine $k$.
        \item SSA.
        \begin{itemize}
            \item If \ce{A} is more stable than \ce{B}, the SSA is valid.
            \item If \ce{I} is depleted faster than it is formed, the SSA is valid.
        \end{itemize}
        \item QEA.
        \begin{itemize}
            \item Only valid when $k_{-1}>k_2$.
        \end{itemize}
        \item Limiting cases.
        \item The saturation regime.
        \item Defining the total concentration of the catalyst is often useful.
        \item Michaelis-Menten kinetics.
        \item Blackmond's work.
        \begin{itemize}
            \item Same-excess experiment and visual overlay.
            \item Different excess experiment.
        \end{itemize}
    \end{itemize}
    \item Kinetic resolution.
    \begin{itemize}
        \item Relations between conversion and ee.
        \item Dynamic kinetic resolution: Going through an intermediate or other interconversion.
        \item Curtin-Hammett kinetics.
    \end{itemize}
\end{itemize}



\section{Electron Transfer}
\begin{itemize}
    \item \marginnote{12/5:}Through the end of the semester, we'll look at classes of reactions we haven't yet discussed.
    \begin{itemize}
        \item Thus far, we've looked at pairwise movements of electrons; we'll now get into single-electron transfer.
        \item Broadly, this is \textbf{open shell chemistry}.
    \end{itemize}
    \item Today: The fundamentals of electron transfer processes.
    \item Three types of electron transfer.
    \begin{enumerate}
        \item Electron transfer: A radical anion (donor) transfers to a neutral acceptor.
        \begin{equation*}
            \ce{D\ra{} + A -> D + A\ra}
        \end{equation*}
        \item Hole transfer: A donor transfers to an electron-deficient radical cation.
        \begin{equation*}
            \ce{D + A\rc{} -> D\rc{} + A}
        \end{equation*}
        \begin{itemize}
            \item We can also think of this as a hole transfer from the acceptor to the donor.
        \end{itemize}
        \item Charge transfer: The donor and acceptor are both neutral, but are induced into a radical cation/anion pair.
        \begin{equation*}
            \ce{D + A -> D\rc{} + A\ra}
        \end{equation*}
    \end{enumerate}
    \item How do we know when any of these electron transfers are spontaneous?
    \begin{itemize}
        \item To answer this question, we need to know something about the free energy of an electron transfer.
        \item Said free energy is related as follows.
        \begin{equation*}
            \Delta G^\circ = -nF\Delta E^\circ
            = -nF(\Delta E^\circ_{\ce{D}}-\Delta E^\circ_{\ce{A}})
        \end{equation*}
        \begin{itemize}
            \item $n$ is the number of electrons transferred. $n=1$ in each of the above 3 examples.
            \item $F=\SI[per-mode=symbol]{96485}{\coulomb\per\mole}$ is Faraday's constant.
            \item $\Delta E^\circ$ is the standard reduction potential for the given species.
            \item Double check this equation --- not sure it's right or what sign convention we're using??
        \end{itemize}
        \item But how do we determine standard reduction potentials?
    \end{itemize}
    \item Standard reduction potentials are determined using electrochemistry, most commonly cyclic voltammetry (CV).
    \pagebreak
    \item Cyclic voltammetry overview.
    \begin{figure}[h!]
        \centering
        \includegraphics[width=0.4\linewidth]{CVsetup.JPG}
        \caption{Cyclic voltammetry setup.}
        \label{fig:CVsetup}
    \end{figure}
    \begin{itemize}
        \item Take your analyte of interest (\ce{D}), dissolve it in a medium of interest, and to make your medium sufficiently conductive, also dissolve in an electrolyte.
        \item In a 2-electrode system (which stands in contrast to some modern 3-electrode systems), bathe in the \textbf{working electrode} and \textbf{auxiliary electrode}.
        \item Connect the two electrodes with a system that will allow us to continuously vary the voltage across those electrodes.
        \item Modify the voltage until such a time as we close the circuit by donating an electron to \ce{D}, making it \ce{D\ra}.
    \end{itemize}
    \item Imagine your electrode as a metal that has a continuum of states up to a point.
    \begin{figure}[h!]
        \centering
        \includegraphics[width=0.5\linewidth]{CVET.JPG}
        \caption{Cyclic voltammetry electron transfer.}
        \label{fig:CVET}
    \end{figure}
    \begin{itemize}
        \item That point is denoted $E_0$.
        \item The donor molecule will have some filled orbitals and empty orbitals. One of the filled orbitals is a HOMO, and the other is a LUMO.
        \item By varying the potential, we can alter the electrochemical potential up, to the point that the electrode can spontaneously populate the donor's LUMO.
    \end{itemize}
    \item Sergei: Why is only one electron transferred to the LUMO if more than one is available?
    \begin{itemize}
        \item The Fermi level will probably be equal to the LUMO.
        \item You can keep pumping electrons in if you apply a higher voltage.
    \end{itemize}
    \pagebreak
    \item What does a cyclic voltammogram look like?
    \begin{figure}[h!]
        \centering
        \includegraphics[width=0.4\linewidth]{CVgraph.JPG}
        \caption{Cyclic voltammogram.}
        \label{fig:CVgraph}
    \end{figure}
    \begin{itemize}
        \item We plot current against voltage, starting at an open circuit potential $V_\text{OCP}:=0$.
        \item As we decrease the voltage, we will essentially reach a point at which current begins being delivered (reduction) and subsequently slows (because of diffusion control).
        \begin{itemize}
            \item The shape rigorously depends on diffusion and more complex math that Alex won't go into.
        \end{itemize}
        \item Then as we scan back forward, we get oxidation.
        \item $E_{1/2}$ is the midpoint of the two peaks, and is the reduction potential for the reaction
        \begin{equation*}
            \ce{D + e^- -> D\ra}
        \end{equation*}
    \end{itemize}
    \item We also need a reference. Most frequently, we reference to the ferrocene/ferrocenium redox couple.
    \begin{equation*}
        \ce{Fc+ + e^- -> Fc}
    \end{equation*}
    \begin{itemize}
        \item Note that $\ce{Fc}\equiv\ce{Cp2Fe^{II}}$.
    \end{itemize}
    \item Kinetics of electron transfer.
    \begin{figure}[h!]
        \centering
        \includegraphics[width=0.6\linewidth]{ETDistort.JPG}
        \caption{Electron transfer can induce geometric distortion.}
        \label{fig:ETDistort}
    \end{figure}
    \begin{itemize}
        \item In this example, we'll consider benzene and its reduction to the corresponding radical anion.
        \item Recall that the frontier orbitals of benzene are given by H\"{u}ckel theory (see Figure \ref{fig:HuckelBenzene}).
        \item Thus, in this reaction, we will drop an electron into a doubly degenerate frontier orbitals.
        \begin{itemize}
            \item The system will then distort to relieve that degeneracy and lower the energy of the overall system; this is called a \textbf{Jahn-Teller distortion}.
            \item This also implies that the benzene radical anion will distort from perfect six-fold symmetry.
        \end{itemize}
        \item It could be a \ce{C-C} bond elongation, but the true distortion's nature is somewhat under debate.
        \begin{itemize}
            \item The actual distortion is likely a slight out of plane ruffling in a fedora/sombrero-type shape.
        \end{itemize}
    \end{itemize}
    \item Let's now take a look at the self-exchange electron transfer reaction.
    \begin{figure}[H]
        \centering
        \begin{subfigure}[b]{\linewidth}
            \centering
            \includegraphics[width=0.45\linewidth]{ETselfExa.JPG}
            \caption{Possible mechanism.}
            \label{fig:ETselfExa}
        \end{subfigure}\\[2em]
        \begin{subfigure}[b]{\linewidth}
            \centering
            \includegraphics[width=0.53\linewidth]{ETselfExb.JPG}
            \caption{Actual mechanism.}
            \label{fig:ETselfExb}
        \end{subfigure}
        \caption{Self-exchange electron transfer.}
        \label{fig:ETselfEx}
    \end{figure}
    \begin{itemize}
        \item Mix benzene (\ce{A}) and the distorted benzene radical anion (\ce{B}).
        \item If this does an electron transfer without any nuclear motion, it will then be followed by thermodynamically downhill nuclear reorganization, but this is in contrast with the law of conservation of energy, so it must not be this mechanism (Figure \ref{fig:ETselfExa}).
        \begin{itemize}
            \item Notice that the forward and reverse reactions also have different mechanisms here, so we are \emph{not} obeying microscopic reversibility.
        \end{itemize}
        \item Real mechanism: \ce{A} elongates and \ce{B} shrinks until we can have an energetically degenerate electron transfer (Figure \ref{fig:ETselfExb}).
        \begin{itemize}
            \item This is a closed thermodynamic square that is net neutral in free energy.
            \item It also obeys the principle of microscopic reversibility!
        \end{itemize}
    \end{itemize}
    \item Self-exchange electron transfer on a free energy surface.
    \begin{figure}[H]
        \centering
        \includegraphics[width=0.5\linewidth]{EDETselfEx.JPG}
        \caption{Self-exchange electron transfer energy diagram.}
        \label{fig:EDETselfEx}
    \end{figure}
    \begin{itemize}
        \item Parabolic shape comes from the Morse potential and its harmonicity near the bottom.
        \item Where the curves cross corresponds to the transition states where electron transfer happens.
        \item This is a thermoneutral, degenerate reaction with $\Delta G_\text{rxn}=0$.
        \item However, we can quantify the barrier $\Delta G^\ddagger$ to self exchange.
        \item A Franck-Condon electron transfer with no nuclear motion is denoted by $\lambda$ and called the \textbf{reorganization energy}.
    \end{itemize}
    \item What about when electron transfer is nondegenerate?
    \begin{figure}[h!]
        \centering
        \includegraphics[width=0.45\linewidth]{EDETnondegen.JPG}
        \caption{Nondegenerate electron transfer energy diagram.}
        \label{fig:EDETnondegen}
    \end{figure}
    \begin{itemize}
        \item Apply the same formalism, but modify it: Simply shift the product parabola down.
        \item To understand the kinetics, we still need to understand $\Delta G^\ddagger$.
        \item We can do this as follows.
    \end{itemize}
    \item Consider the parabolas existing in a Cartesian plane, with the origin at the vertex of the starting material parabola (see Figure \ref{fig:EDETnondegen}).
    \begin{itemize}
        \item Then the product parabola has vertex at $(a,b)$.
        \item It follows from a bit of algebra that
        \begin{align*}
            x^2 &= (x-a)^2+b\\
            x &= \frac{b+a^2}{2a}
        \end{align*}
        so the parabolic curve crossing happens at
        \begin{equation*}
            \left( \frac{b+a^2}{2a},\frac{(b+a^2)^2}{4a^2} \right)
        \end{equation*}
        \item Additionally, $b$ is the $\Delta G$ of the reaction, and $a^2$ is the reorganization energy $\lambda$.
        \item Thus,
        \begin{equation*}
            \Delta G^\ddagger = \frac{(\Delta G+\lambda)^2}{4\lambda}
        \end{equation*}
        \begin{itemize}
            \item This is the \textbf{Marcus relationship}.
        \end{itemize}
    \end{itemize}
    \pagebreak
    \item Example cases of electron transfer with varied driving force.
    \begin{figure}[h!]
        \centering
        \begin{subfigure}[b]{0.3\linewidth}
            \centering
            \includegraphics[width=0.95\linewidth]{ETforcea.JPG}
            \caption{No driving force.}
            \label{fig:ETforcea}
        \end{subfigure}
        \begin{subfigure}[b]{0.3\linewidth}
            \centering
            \includegraphics[width=0.93\linewidth]{ETforceb.JPG}
            \caption{A driving force.}
            \label{fig:ETforceb}
        \end{subfigure}\\[2em]
        \begin{subfigure}[b]{0.3\linewidth}
            \centering
            \includegraphics[width=0.9\linewidth]{ETforcec.JPG}
            \caption{Maximum driving force.}
            \label{fig:ETforcec}
        \end{subfigure}
        \begin{subfigure}[b]{0.3\linewidth}
            \centering
            \includegraphics[width=0.9\linewidth]{ETforced.JPG}
            \caption{Inverted driving force.}
            \label{fig:ETforced}
        \end{subfigure}
        \caption{Electron transfer with different driving forces.}
        \label{fig:ETforce}
    \end{figure}
    \begin{enumerate}
        \item We could have no driving force.
        \item We could have a driving force.
        \item We could have a driving force so great that the transfer is barrierless.
        \item We could have a driving force so great that we begin to get a barrier again!
    \end{enumerate}
    \item Electron transfer driving forces induce varying rates.
    \begin{figure}[h!]
        \centering
        \includegraphics[width=0.4\linewidth]{ETforceRate.JPG}
        \caption{Electron transfer rate with different driving forces.}
        \label{fig:ETforceRate}
    \end{figure}
    \begin{itemize}
        \item As we increase the driving force, we would decrease the barrier $\Delta G^\ddagger$.
        \item Thus, the rate should increase with increasing driving force.
        \item The rate should increase until $\Delta G_\text{rxn}=\lambda$, where the product parabola intersects the starting material vertex.
        \item If we try to go faster than this, we have a driving force so profound that we build back in a barrier to the electron transfer.
        \item However, practically, we reach the diffusional limit in rate.
    \end{itemize}
    \item Discovering the Marcus inverted region.
    \begin{figure}[H]
        \centering
        \begin{subfigure}[b]{0.45\linewidth}
            \centering
            \includegraphics[width=0.65\linewidth]{marcusInva.JPG}
            \caption{A designer molecule.}
            \label{fig:marcusInva}
        \end{subfigure}
        \begin{subfigure}[b]{0.45\linewidth}
            \centering
            \includegraphics[width=0.95\linewidth]{marcusInvb.JPG}
            \caption{Varying the acceptor.}
            \label{fig:marcusInvb}
        \end{subfigure}
        \caption{Experimental design of Marcus inverted electron transfer.}
        \label{fig:marcusInv}
    \end{figure}
    \begin{itemize}
        \item Rudy Marcus made this theoretical prediction 50-60 years ago.
        \item People tried to prove it by mixing very strong reductants and oxidants, but there was a flaw in this experimental setup: Diffusion control, aka the diffusional limit for bimolecular electron transfer.
        \item Indeed, we needed to eliminate diffusion control, which we did by looking at unimolecular processes.
        \item We took a biphenyl species, loaded it with an electron (as in Figure \ref{fig:ETselfEx}), and then studied it with pulse radiolysis (where we attach an electron onto the biphenyl with high-energy pulsed electrons).
        \item The biphenyl was attached onto a rigid, steroidal hydrocarbon. This hydrocarbon is a linker of defined length, and then we appended an acceptor to the other end.
        \item The acceptor was varied to change the driving force over about \kcal{40} (which is about \SI{2}{\electronvolt}).
        \item Naphthalene is a weak acceptor, then anthracene is a bit better, then pyrene.
        \item Then quinones, and halogenated quinones could finally get us to the Marcus inverted region.
        \item This is a landmark paper in electron transfer theory: \textcite{bib:marcusInv}.
        \item These donors and acceptors had likely all been explicitly studied in an intermolecular sense.
        \begin{itemize}
            \item Diffusion limit might have slowed reactions before the scenario in Figure \ref{fig:ETforcec}.
        \end{itemize}
        \item A lot of pulse radiolysis is done cryogenically, so diffusion was likely entirely taken out of consideration.
    \end{itemize}
    \item Next class: Photophysics, the photochemistry of excited states, etc.
\end{itemize}




\end{document}