\documentclass[../notes.tex]{subfiles}

\pagestyle{main}
\renewcommand{\chaptermark}[1]{\markboth{\chaptername\ \thechapter\ (#1)}{}}
\setcounter{chapter}{2}

\begin{document}




\chapter{???}
\section{Computational Chemistry}
\begin{itemize}
    \item \marginnote{9/17:}Lecture 3 recap.
    \begin{itemize}
        \item Huckel theory: A fast way to draw the MOs of conjugated $\pi$-systems.
        \begin{itemize}
            \item If the conjugated $\pi$-system in question is cyclic, use a Frost circle.
        \end{itemize}
        \item Aromaticity.
        \begin{itemize}
            \item Huckel's definition: $4n+2$.
            \item M\"{o}bius's definition: $4n$.
            \item Leads to properties like stabilization, quadrupoles, and ring current.
        \end{itemize}
        \item Cyclopropane: $sp^2$-like banana bonds (the only thing we need to remember from that discussion).
        \item Wavefunctions: Solutions to the Schr\"{o}dinger equation.
    \end{itemize}
    \item Today: Computational chemistry (an overview).
    \begin{itemize}
        \item Computational chemistry is typically an entire class!
    \end{itemize}
    \item Lecture outline.
    \begin{itemize}
        \item Methods of computational chemistry.
        \item Molecular mechanics.
        \item Semi-empirical methods.
        \item Ab initio methods.
        \item Hartree-Fock.
        \item Density functional theory (DFT).
        \item Best practices for calculations.
        \item Properties that are especially easy (or hard) to calculate.
    \end{itemize}
    \item Why do we do computational chemistry?
    \begin{itemize}
        \item If we could fully solve the Schr\"{o}dinger equation, we could know the properties of all of our electrons!
        \item However, the Schr\"{o}dinger equation can only be fully solved (practically) for the simplest systems.
        \begin{itemize}
            \item For now, at least: People are working on this.
        \end{itemize}
        \item As such, we \emph{approximate} solutions instead.
    \end{itemize}
    \pagebreak
    \item \textbf{Computational chemistry}: The science of approximating solutions to the Schr\"{o}dinger equation.
    \begin{itemize}
        \item Computational chemistry can be broken up into two general strategies (\textbf{ab initio} and \textbf{empirical} methods) and one in-between strategy called \textbf{semi-empirical} mehods.
    \end{itemize}
    \item \textbf{Ab initio} (methods): Make well-defined approximations to the Schr\"{o}dinger equation, and then solve the approximations mathematically. \emph{Etymology} from Latin "from first principles."
    \begin{itemize}
        \item Essentially, make your math simpler.
    \end{itemize}
    \item \textbf{Semi-empirical} (methods): Replace complicated parts of the Schr\"{o}dinger equation with experimentally derived parameters, such as bond lengths, vibrational frequencies, and more that we can get from spectroscopy.
    \begin{itemize}
        \item Essentially, shortcut the hardest parts of solving with experimentally derived features.
    \end{itemize}
    \item \textbf{Empirical} (methods): Approximate molecules with force fields that are experimentally derived, and adjust with further experimental parameters.
    \begin{itemize}
        \item Essentially, start with reality and derive computational things from that.
    \end{itemize}
    \item We now look at some commonly derived methods. The following list is sorted from methods with high \textbf{accuracy} and low \textbf{speed} to methods with low accuracy and high speed.
    \begin{itemize}
        \item Methods at the high end of accuracy and the low end of speed (ab initio).
        \begin{itemize}
            \item \textbf{Coupled cluster}.
            \item \textbf{Perturbation theory}.
            \item \textbf{Density functional theory}.
            \item \textbf{Hartree-Fock}.
        \end{itemize}
        \item Methods in the middle (semi-empirical).
        \begin{itemize}
            \item \textbf{Semi-empirical methods}.
        \end{itemize}
        \item Methods at the high end of speed and the low end of accuracy (empirical).
        \begin{itemize}
            \item \textbf{Molecular mechanics}.
        \end{itemize}
    \end{itemize}
    \item \textbf{Speed}: Ease of calculations.
    \item \textbf{Accuracy}: Careful and diligent.
    \item \textbf{Coupled cluster}: Useful for approximately 10 \textbf{heavy atoms}. \emph{Also known as} \textbf{CC}.
    \item \textbf{Density functional theory}: Useful for approximately 80 heavy atoms, though we can use more (it just gets slower). \emph{Also known as} \textbf{DFT}.
    \item \textbf{Hartree-Fock}. \emph{Also known as} \textbf{HF}.
    \item \textbf{Molecular mechanics}: Useful for hundreds of heavy atoms. \emph{Also known as} \textbf{MM}.
    \item \textbf{Heavy atom}: Any atom that's not hydrogen.
    \item In this course, we'll discuss further the bottom four methods in the above list of six.
    \item Molecular mechanics (MM).
    \begin{itemize}
        \item Atoms are treated as balls and springs (this is a classical analogy and thus much easier to simulate).
        \item We use force fields to describe electrons.
        \begin{itemize}
            \item These force fields are derived from experimental data, i.e., choose a force field that gives us the bonds we calculate from XRD or the vibrations we see in IR.
        \end{itemize}
        \item Very fast; often considered "quick and dirty."
        \begin{itemize}
            \item Gives us a general picture of what we're thinking about.
        \end{itemize}
        \item Common application: Very large and flexible systems.
        \begin{itemize}
            \item Think proteins, polymers, etc.
            \item Things that have a lot of degrees of freedom.
            \item Very useful for chembio, polymer chemistry, etc.
        \end{itemize}
        \item Subset application: \textbf{Molecular dynamics} (MD).
        \begin{itemize}
            \item Simulating movement; uses MM as a basis.
        \end{itemize}
        \item If you're going to use this method, know that it is (in general) only appropriate for approximating the ground states of molecules (not their transition states).
        \begin{itemize}
            \item However, MM can be a good starting point for higher-level calculations (i.e., more accurate methods).
            \item In Orgo, it's mainly used for first approximations to be refined later (and for heavier stuff).
            \item All the same, it is a super useful tool with tons of applications, and its simplicity should not lead us to discount it.
        \end{itemize}
        \item Running MM.
        \begin{itemize}
            \item If we have a PC, try clicking the MM2 button in Chem3D (which is part of our ChemDraw package).
            \item This may not work on Macs; figure this out!!
            \item PerkinElmer (who developed ChemDraw) initially developed their stuff for Macs; Masha's not quite sure where they dropped the ball.
        \end{itemize}
    \end{itemize}
    \item Semi-empirical quantum mechanical (SQM) methods.
    \begin{itemize}
        \item Use empirical parameters to simplify \emph{ab initio} calculations.
        \begin{itemize}
            \item Tries to deliver the best of both worlds (speed and accuracy).
        \end{itemize}
        \item We can add corrections for missing phenomena and underestimated features.
        \item Theoreticians (developers) will draw the line on accuracy somewhere, and then organic chemists will say, "this model fails here."
        \begin{itemize}
            \item Once that feedback gets into the literature, theoreticians redefine their line.
            \item They might need to account for $d$-orbitals, London dispersion forces (LDFs), flexibility, solvent, or more.
            \begin{itemize}
                \item Methods of accounting for solvent effects are continuously being optimized.
            \end{itemize}
            \item It's important to be on top of the literature here, since things are always getting better!
        \end{itemize}
        \item Modern implementations (these are getting fast enough to be usable and really good!).
        \begin{itemize}
            \item Density function based tight binding (DFTB): Approximate DFT.
            \item eXtended Tight Binding (xTB)
            \begin{itemize}
                \item Developed primarily by the Grimme lab.
                \item Basically just adding more parameters.
            \end{itemize}
        \end{itemize}
        \item LDFs are becoming increasingly important for selective catalysis, so there's a lot of work to approximate them.
        \begin{itemize}
            \item Catalysis is not about partial positive and negative charges so much as it is about electrons flopping around to achieve incredible selectivities in next-gen catalysts.
        \end{itemize}
        \item Very fast (seconds) and pretty accurate. Increasingly used, especially for ML and data science.
        \begin{itemize}
            \item Nowadays, if you want to do ML, you need these hundreds of experimental data points.
        \end{itemize}
    \end{itemize}
    \item Ab initio methods.
    \begin{itemize}
        \item Background theories (neither is technically true, but it is helpful for speed).
        \begin{itemize}
            \item \textbf{Born-Oppenheimer approximation}.
            \item \textbf{Independent electron theory}.
        \end{itemize}
    \end{itemize}
    \item \textbf{Born-Oppenheimer approximation}: Nuclei are way bigger than electrons (have over 1000 times more mass), so they are basically fixed in space relative to the electrons.
    \begin{itemize}
        \item This means that you can treat the nuclei separately; you can use one approach for the nuclei and an entirely different approach for the electrons.
    \end{itemize}
    \item \textbf{Independent electron theory}: Electron movements are not correlated to each other; all electrons whiz around independently.
    \begin{itemize}
        \item Making this approximation will cause some issues.
    \end{itemize}
    \item Hartree-Fock (HF).
    \begin{itemize}
        \item Treat electrons as a delocalized cloud with independent electron movement.
        \begin{itemize}
            \item Remember the plum pudding model of the atom? This is not that dissimilar from that.
        \end{itemize}
        \item This approximation ignores Coulombic interactions (like LDFs).
        \begin{itemize}
            \item This becomes very problematic for transition states.
        \end{itemize}
        \item HF methods are largely historical today.
        \begin{itemize}
            \item There are applications where they're still used today, but not in Orgo and not without an understanding of their shortcomings.
        \end{itemize}
    \end{itemize}
    \item We can run any and all of these computations throughout grad school as MIT students, and we should! They're in our toolbelt now, and we should try them out!!
    \item Density functional theory (DFT).
    \begin{itemize}
        \item Instead of calculating wavefunctions, we're going to calculate electron density.
        \begin{itemize}
            \item We're going to do this using \textbf{functionals}.
            \item We can include functionals for things like Coulombic interactions, etc.\footnote{Maybe what I can be known for in research is custom building computational tools for specific organic problems, and turning that into a workflow that people do. Maybe that's what ML already is.}
        \end{itemize}
        \item This is a good workhorse method in organic chemistry.
        \begin{itemize}
            \item DFT is appropriate for reaction coordinate mapping, transition states, etc.
            \item We'll often work with collaborators that can tailor a model to our needs.
        \end{itemize}
        \item There are many specific functionals and basis sets.
        \begin{itemize}
            \item You have to choose the functional (choose what to include), and then choose the basis sets (how much detail do I need for this calculation, e.g., treating polarization, charge, unpaired electrons more accurately).
            \item It is best to find a basis set and functional appropriate for our context.
        \end{itemize}
        \item Basis sets don't describe all types of elements.
        \begin{itemize}
            \item Some describe elements 1-30, others do 1-86.
            \item Don't be that person who has to redo their entire calculation because they forgot that tin is one of their reagents!
            \item We often use \textbf{split basis sets} (esp. for transition metals), i.e., certain atoms (i.e., metals) get more functionals.
            \begin{itemize}
                \item Carbon, hydrogen, and oxygen (CHO) don't need the craziest level of theory to approximate, but that palladium center will!
            \end{itemize}
            \item Think about what level of theory you need for each atom.
        \end{itemize}
    \end{itemize}
    \item \textbf{Functional}: A function of functions.\footnote{This is the computer science definition; it is largely unrelated to the mathematical definition that is equivalent to linear forms and duals.} \emph{Also known as} \textbf{higher-order function}.
    \item Best practices for running calculations for our own things.
    \begin{itemize}
        \item This part of the lecture is \emph{critical}; it tells us what we need to know to use computational chemistry.
        \begin{itemize}
            \item If we want to learn the theory for all of these things, we should read a textbook or take Heather Kulik's class.
        \end{itemize}
        \item Use the appropriate level of theory for your needs and capabilities.
        \item Questions to ask yourself to assess your needs and capabilities.
        \begin{itemize}
            \item Do you have a supercomputer? How much time on the supercomputer do you have?
            \item When do you need this result by? Is your PI breathing down your neck?
            \item What am I trying to model?
            \item Is this a thought experiment or something serious?
        \end{itemize}
        \item Additional things to consider wrt your needs and capabilities.
        \begin{itemize}
            \item Consider speed vs. accuracy.
            \begin{itemize}
                \item You can always start at a lower level theory and then ramp it up if you need more accuracy. This is a great general approach.
            \end{itemize}
            \item Consider size and flexibility (no HF on proteins, or MM on methane).
            \item Consider "weirdness": If you've got something that's all inverted and M\"{o}bius like, you're gonna need something more tailor-made.
            \item Find a \emph{reliable} literature precedent for a similar system.
            \begin{itemize}
                \item If you want to model a cationic cyclization, use a precedent paper's level of theory.
                \item How do I model an iridium catalyst? Find an iridium catalyst paper and go from that!
            \end{itemize}
            \item Know how your level of theory works.
            \begin{itemize}
                \item Does it account for polarizability? Charge? Solvent? $d$-orbitals?
                \item It is our responsibility as an experimentalist to know this if we're going to publish it; our PI probably won't be as deep into the nitty-gritty as us.
            \end{itemize}
        \end{itemize}
        \item Don't blindly trust calculations.
        \begin{itemize}
            \item Calculations always give you an answer (unless they fail or don't converge). However, just because you get a number doesn't means that that number is accurate!
            \item Benchmark your calculation with experiments whenever we can. Examples: X-ray structure, ratio of products (we can back-calculate from temperature the activation energy barrier, and from the transition states what ratio of products we expect).
            \item Redo a couple of calculations at a higher level of theory to see if you get the same answer.
            \item Chris Cramer (a founding father of computational chemistry): "There is no particular virtue to the speed at which a wrong answer can be obtained."
        \end{itemize}
        \item Example of doing calculations wrong: Doing an S\textsubscript{N}1 reaction without solvent. These reactions are so solvent-dependent, and there's no gas-phase cation that will replicate this solution-phase reaction.
    \end{itemize}
    \item What's easy to calculate?
    \begin{itemize}
        \item Spectra: IR, Raman, NMR.
        \begin{itemize}
            \item Masha likes to predict the NMR spectra of wacky intermediates.
            \item ChemDraw does this for free.
            \begin{itemize}
                \item The default solvent is THF; make sure you change it to \ce{CDCl3}!
            \end{itemize}
            \item MNova's function is better; it's ML-based, but it also costs money to run?? I think Masha has this wrong for MIT students.
        \end{itemize}
        \item Geometries, conformers, and ground state structures.
        \begin{itemize}
            \item "Geometry optimization" or "energy minimization" is very common.
            \item Draw a 3D structure, give it to our program, move atoms, calculate $E$, repeat (let the program perturb the atom's positions a bit) until we reach a \emph{local} minimum.
            \item This is what we'll do on the problem set.
            \item If we want to get the \emph{global} minimum, we have to look for lower energy structures (manually, automatically, or a combination of both).
            \item We often start at a low level of theory and then refine. Start with a search of the chemical space to find some stable conformers, and then pop that into DFT.
        \end{itemize}
        \item Frequencies, well-defined transition states, and \textbf{single point calculations}.
        \begin{itemize}
            \item Important because transition states are saddle points on the potential energy surface with 1 imaginary frequency corresponding to the bond-making or -breaking event.
            \item If you have a structure that you think is a ground state, you have to prove this.
        \end{itemize}
    \end{itemize}
    \item \textbf{Single point calculation}: Calculating the energy of a structure without any other atoms around.
    \item Note: Nucleophiles typically come in at a \ang{120} angle (the \textbf{B\"{u}rgi-Dunitz angle}), because that's where it's easiest to donate into the $\pi^*$-lobe.
    \item What's "hard" to calculate?
    \begin{itemize}
        \item Caveat: Do your research!!
        \begin{itemize}
            \item Many applications require specialized approaches.
            \item There's an army of computational chemists who are trying to develop niche methods for our little problem; find them, connect with them, collaborate with them, etc.
            \item It is our responsibility to know what part of a certain experiment is difficult.
            \begin{itemize}
                \item Example: In photophysics, you need to know the limitations of certain parts of our model.
                \item The system will not say, "I'm bad at predicting excited states;" you have to know that.
            \end{itemize}
        \end{itemize}
        \item Things that require more finessing to study.
        \begin{itemize}
            \item Open-shell species, e.g., radicals.
            \item Transition metals: Heather researches how to model TMs with SQM, etc. This is really important and really hard.
            \item "Unusual structures," e.g., gas-phase plasma nonsense.
        \end{itemize}
        \item Thermochemistry.
        \begin{itemize}
            \item E.g., thermodynamical parameters, calorimetry, etc.
            \item There are specific packages that work for this, if we need to look into them.
            \item Masha doesn't know anything about any of this, but recommends that we can learn them!!
        \end{itemize}
        \item Solvent effects, LDFs, etc.
        \begin{itemize}
            \item For these topics, the methods are getting better all the time (which is code for, "the programs don't work great yet").
            \item \textbf{Implicit solvation} vs. \textbf{explicit solvation}.
        \end{itemize}
    \end{itemize}
    \item \textbf{Implicit solvation}: Treat the solvent as a continuous medium.
    \item \textbf{Explicit solvation}: Draw solvent molecules and add them to the calculation.
    \begin{itemize}
        \item If you think the solvent is stabilizing the transition state in an S\textsubscript{N}1, you need to draw a little THF donating its lone pair to the carbocation.
        \item This gets complicated in proteins, where we need to identify how many waters we need to draw to accurately represent water; recent papers suggest that effects matter up to 44 \ce{H2O} molecules away!
    \end{itemize}
\end{itemize}




\end{document}