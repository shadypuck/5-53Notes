\usepackage[margin=1in]{geometry}
\usepackage{csquotes}
\usepackage{fancyhdr}
\usepackage{marginnote}
\usepackage{enumitem}
\usepackage{scrextend}
\usepackage[bottom]{footmisc}
\usepackage{siunitx}
\usepackage[style=apa]{biblatex}
\usepackage{xr}
\usepackage{tikz,graphicx}
\usepackage{subcaption,float}
\usepackage{amsmath,amssymb,amsthm}
\usepackage{physics,bm,xfrac}
\usepackage{mhchem}
\usepackage{chemfig}
\usepackage[colorlinks,allcolors=black,urlcolor=cyan]{hyperref}
\usepackage{subfiles}

\MakeOuterQuote{"}

\fancypagestyle{main}{
    \fancyhf{}
    \fancyhead[L]{\leftmark}
    \fancyhead[R]{5.53}
    \fancyfoot[R]{Labalme\ \thepage}
}
\fancypagestyle{plain}{
    \fancyhead{}
    \renewcommand{\headrulewidth}{0pt}
}

\reversemarginpar

\setitemize[3]{label={\scriptsize$\blacksquare$}}
\setitemize[4]{label={\tikz[scale=0.06,baseline={(0,-0.14)}]{
    \draw [line width=0.3pt] (0,1) -- (1.2,0) -- (0,-1) -- (3.5,0) -- cycle;
    \fill (1.2,0) -- (0,-1) -- (3.5,0);
}}}

\deffootnotemark{\textsuperscript{\textup{[}\thefootnotemark\textup{]}}}
\deffootnote[2.1em]{0em}{0em}{\textsuperscript{\thefootnote}}

\sisetup{range-phrase=-,range-units=single}
\AtBeginDocument{\RenewCommandCopy\qty\SI}
\DeclareSIUnit{\calorie}{cal}

\addbibresource{\subfix{../main.bib}}
\DefineBibliographyStrings{english}{bibliography={References}}

\usetikzlibrary{bending,calc}
\tikzset{
    curved arrow/.style 2 args={magenta,semithick,shorten <=#1,shorten >=#2}
}
\colorlet{grx}{green!50!black}
\colorlet{orx}{orange!90!yellow!95!black!90}
\colorlet{ory}{orange!90!yellow!95!black!20}

\graphicspath{{../ExtFiles/}}

\setchemfig{atom sep=2em,fixed length=true,bond offset=3pt,cram width=3pt}
\setcharge{extra sep=3pt}
\catcode`\_=11
\definearrow3{=retro>}{%
    \CF_arrowshiftnodes{#3}%
    \draw[double distance=2pt, -Implies] (\CF_arrowstartnode)--(\CF_arrowendnode);
    \expandafter[\CF_arrowcurrentstyle](\CF_arrowstartnode)--(\CF_arrowendnode);%
    \CF_arrowdisplaylabel{#1}{0.5}+\CF_arrowstartnode{#2}{0.5}-\CF_arrowendnode
}
\catcode`\_=8

\newcommand{\e}[1][]{\textup{e}^{#1}}
\newcommand{\cnc}[1]{[\ce{#1}]}
\newcommand{\pKa}{\textup{p}K_\textup{a}}
\newcommand{\Ka}{K_\textup{a}}