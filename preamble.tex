\usepackage[margin=1in]{geometry}
\usepackage{csquotes}
\usepackage{fancyhdr}
\usepackage{tocloft}
\usepackage{marginnote}
\usepackage{enumitem}
\usepackage{scrextend}
\usepackage[bottom]{footmisc}
\usepackage{siunitx,xparse}
\usepackage[style=apa]{biblatex}
\usepackage{xr}
\usepackage{tikz,graphicx}
\usepackage{subcaption,float}
\usepackage{amsmath,amssymb,amsthm}
\usepackage{physics,bm,xfrac,mathtools}
\usepackage{mhchem}
\usepackage{chemfig}
\usepackage[colorlinks,allcolors=black,urlcolor=cyan]{hyperref}
\usepackage{subfiles}

\MakeOuterQuote{"}

\fancypagestyle{main}{
    \fancyhf{}
    \fancyhead[L]{\leftmark}
    \fancyhead[R]{5.53}
    \fancyfoot[R]{Labalme\ \thepage}
}
\fancypagestyle{plain}{
    \fancyhead{}
    \renewcommand{\headrulewidth}{0pt}
}

\reversemarginpar

\setitemize[3]{label={\scriptsize$\blacksquare$}}
\setitemize[4]{label={\tikz[scale=0.06,baseline={(0,-0.14)}]{
    \draw [line width=0.3pt] (0,1) -- (1.2,0) -- (0,-1) -- (3.5,0) -- cycle;
    \fill (1.2,0) -- (0,-1) -- (3.5,0);
}}}

\deffootnotemark{\textsuperscript{\textup{[}\thefootnotemark\textup{]}}}
\deffootnote[2.1em]{0em}{0em}{\textsuperscript{\thefootnote}}

\sisetup{range-phrase=-,range-units=single}
\AtBeginDocument{\RenewCommandCopy\qty\SI}
\NewDocumentCommand\angrange{O{} m m}{\SIrange[parse-numbers=false, #1]{\ang[parse-numbers=true]{#2}}{\ang[parse-numbers=true]{#3}}{}}
\DeclareSIUnit{\calorie}{cal}
\DeclareSIUnit{\debye}{D}
\DeclareSIUnit{\revolutionsperminute}{rpm}
\DeclareSIUnit{\entropyunit}{e.u.}
\DeclareSIUnit{\molar}{M}
\DeclareSIUnit{\year}{y}

\addbibresource{\subfix{../main.bib}}
\DefineBibliographyStrings{english}{bibliography={References}}

\usetikzlibrary{bending,calc,decorations.pathmorphing,intersections}
\tikzset{
    curved arrow/.style 2 args={magenta,semithick,shorten <=#1,shorten >=#2},
    ovbnd/.style={white,very thick,double=black,double distance=0.4pt},
    wv/.style={decorate,decoration={coil,aspect=0,amplitude=1pt,segment length=4.7pt}}
}
\colorlet{grx}{green!50!black}
\colorlet{gry}{green!50!black!40}
\colorlet{orx}{orange!90!yellow!95!black!90}
\colorlet{ory}{orange!90!yellow!95!black!20}
\colorlet{blx}{blue!90!green!80}
\colorlet{rex}{red!80!black!90!orange!80}

\graphicspath{{../ExtFiles/}}

\DeclareMathOperator{\ZPE}{ZPE}
\DeclareMathOperator{\KIE}{KIE}
\DeclareMathOperator{\rate}{rate}

\setchemfig{atom sep=2em,fixed length=true,bond offset=3pt,cram width=3pt}
\setcharge{extra sep=3pt}
\catcode`\_=11
\definearrow3{=retro>}{%
    \CF_arrowshiftnodes{#3}%
    \draw[double distance=2pt, -Implies] (\CF_arrowstartnode)--(\CF_arrowendnode);
    \expandafter[\CF_arrowcurrentstyle](\CF_arrowstartnode)--(\CF_arrowendnode);%
    \CF_arrowdisplaylabel{#1}{0.5}+\CF_arrowstartnode{#2}{0.5}-\CF_arrowendnode
}
\definearrow0{-->}{
    \edef\dis{1mm}
    \draw [-{Stealth[round,scale=1.1,inset'=0pt 0.55]}] ([xshift=-\dis,yshift=\dis]\CF_arrowstartnode.center) -- ([xshift=-\dis,yshift=\dis]\CF_arrowendnode.center);
    \draw [-{Stealth[round,scale=1.1,inset'=0pt 0.55]}] ([xshift=\dis,yshift=-\dis]\CF_arrowstartnode.center) -- ([xshift=\dis,yshift=-\dis]\CF_arrowendnode.center);
    \expandafter[\CF_arrowcurrentstyle](\CF_arrowstartnode)--(\CF_arrowendnode);
}
\catcode`\_=8
\definesubmol{wave}{-[::180,0.2,,,opacity=0]-[::90,0.5,,,opacity=0]-[::180,,,,wv]}

\newcommand{\e}[1][]{\textup{e}^{#1}}
\newcommand{\cnc}[2][]{[\ce{#2}]_\text{#1}}
\newcommand{\pKa}{\textup{p}K_\textup{a}}
\newcommand{\Ka}{K_\textup{a}}
\newcommand{\pH}{\textup{pH}}
\newcommand{\rc}{
    {\tikz{\node[inner sep=0pt,label={[below=6.5pt,circle,fill,inner sep=0.1pt]},label={[below=7pt,circle,inner sep=0.06pt]}]{\scriptsize $+$};}}
}
\newcommand{\ra}{
    {\tikz{\node[inner sep=0pt,label={[below=4pt,circle,fill,inner sep=0.1pt]},label={[below=7pt,circle,inner sep=0.06pt]}]{\scriptsize $-$};}}
}
\newcommand{\rca}{
    {\tikz{\node[inner sep=0pt,label={[below=6.5pt,circle,fill,inner sep=0.1pt]},label={[below=7pt,circle,inner sep=0.06pt]}]{\scriptsize $\pm$};}}
}
\newcommand{\Keq}{K_\textup{eq}}
\newcommand{\kcal}[1]{\SI[per-mode=symbol]{#1}{\kilo\calorie\per\mole}}
\newcommand{\kcalr}[2]{\SIrange[per-mode=symbol]{#1}{#2}{\kilo\calorie\per\mole}}
\newcommand{\eu}[1]{\SI{#1}{\entropyunit}}
\newcommand{\krel}{k_\textup{rel}}
\newcommand{\kB}{k_\textup{B}}
\newcommand{\kobs}{k_\textup{obs}}
\newcommand{\kf}{k_\textup{fast}}
\newcommand{\ks}{k_\textup{slow}}